%%%%%%%%%%%%%%%%%%%%%%%%%%
\documentclass[letterpaper,12pt]{article}
%%%%%%%%%%%%%%%%%%%%%%%%%%%%%%%%%%%%%%%%%%%%%%%%%%%%%%%%%%%%%%%%%%%%%%%%%%%%%%%%%%%
\usepackage{amsmath}
%\usepackage{etex}
\usepackage{natbib}
\usepackage{amsfonts}
%\usepackage{mathpazo}
\usepackage{hyperref}
\usepackage{xcolor}
\usepackage{multimedia}
\usepackage{graphicx, color}
\usepackage{epsfig}
\usepackage{amsthm}
\usepackage{mathtools}
\usepackage{esint}
\usepackage{amssymb}
\usepackage{url}
\usepackage{relsize}
\usepackage{amsfonts}
%\usepackage{fancyheadings}
\usepackage{float}
\usepackage{color}
\usepackage{mathrsfs}
\usepackage{setspace}
\usepackage[mathscr]{euscript}
\usepackage{caption}
\usepackage{subcaption}
\usepackage{pdflscape}
\usepackage{booktabs}
\usepackage[utf8]{inputenc}
\usepackage[T1]{fontenc}
\usepackage{geometry}
\usepackage{multicol}
\usepackage{multirow}
\usepackage[bottom]{footmisc}
\usepackage{rotating}
\usepackage{verbatim}
\usepackage{dcolumn}
\usepackage{threeparttable}

\setcounter{MaxMatrixCols}{10}

\newcolumntype{d}[1]{D{.}{.}{#1}}
\hypersetup{
    colorlinks,
    linkcolor={blue!75!black},
    citecolor={blue!75!black},
}
%\pdfminorversion 4
\newcommand{\dkblue}{\textcolor[rgb]{.2,.2,.6} }
\newcommand{\redGV}{\textcolor[rgb]{0.89,0.08,0.24}}
\newcommand{\greenGV}{\textcolor[rgb]{0,.5,0}}
\newcommand{\grey}{\textcolor[rgb]{0.29 0.29 0.29}}
\newcommand{\blackGV}{\textcolor[rgb]{0 0 0}}
\newcommand{\bluep}{\textcolor[rgb]{0.0 0.53 0.68}}
\newcommand{\hotpink}{\textcolor[rgb]{0.75,0,0.35}}
\newcommand{\blueGV}{\textcolor[rgb]{0.0 0 0.7}}
\newcommand{\whiteGV}{\textcolor[rgb]{1,1,1}}
\newcommand{\bi}{\begin{itemize}}
\newcommand{\ei}{\end{itemize}}
\newcommand{\ben}{\begin{enumerate}}
\newcommand{\een}{\end{enumerate}}
\definecolor{StataDarkBlue}{rgb}{0,0,0.835}
\newcommand{\jemph}[1]{{\color{StataDarkBlue}#1}}
\newcommand{\jbemph}[1]{\textbf{\color{StataDarkBlue}#1}}
\DeclareMathOperator{\mpc}{mpc}
\providecommand{\keywords}[1]{\textbf{Keywords:}  #1}
\providecommand{\jelclass}[1]{\textbf{JEL codes:}  #1}
\setlength{\topmargin}{-0.4in}
\setlength{\textheight}{8.85in}
\setlength{\oddsidemargin}{-0.2in}
\setlength{\evensidemargin}{0.0in}
\setlength{\textwidth}{7.01in}
\renewcommand{\baselinestretch}{1.0}
\setcounter{page}{1}


%\input{AR_preamble_2.tex}
%%%%%%%%%%%%%%%%%%%%%%%%%%

%\newcommand{\plotdir}{C:/Users/glv2/Dropbox/AnnualReviewsMPC}
\newcommand{\plotdir}{/Volumes/FILES/Dropbox/Projects/AnnualReviewsMPC}

\begin{document}

\title{\textbf{The Marginal Propensity to Consume \\
\vspace*{.25cm}in Heterogeneous Agent Models} \thanks{
We thank Chase Abram and Brian Livingston for outstanding research
assistance.}}
\bigskip
\bigskip

\author{Greg Kaplan\thanks{University of Chicago and NBER: \href{mailto:gkaplan@uchicago.edu}{\texttt{gkaplan@uchicago.edu}}} 
\hspace*{.2cm} and \hspace{.2cm} 
Giovanni L. Violante\thanks{Princeton University, CEBI, CEPR, IFS, IZA, and NBER: \href{mailto:violante@princeton.edu}{\texttt{violante@princeton.edu}}}}

\bigskip
\maketitle

\begin{abstract}
\begin{singlespace}\noindent
We conduct a systematic analysis of partial equilibrium heterogeneous agent consumption models to understand whether and how they can generate a large average marginal propensities to consume (MPC). One-asset models without ex-ante heterogeneity feature a severe trade-off between a high average MPC and a realistic level of aggregate wealth. One-asset models with additional heterogeneity in preferences or rates of return, or behavioral features, can generate high MPCs with the right amount of wealth, but at the cost of an excessively polarized wealth distribution that vastly understates the wealth held by households in the middle of the distribution. Two-asset models that include both liquid and illiquid assets can resolve these trade-offs even without the addition of ex-ante heterogeneity or behavioral elements, however the addition of these features can improve the fit of the model in other dimensions. Across all models, the share and type of hand-to-mouth households is the most important factor that determines the level of the average MPC.
\end{singlespace}
\end{abstract}

\bigskip
\bigskip

\noindent \textbf{Keywords:} Borrowing Constraints, Consumption, Liquidity,
Hand-to-Mouth, Heterogeneity, Marginal Propensity to Consume, Market Incompleteness, Wealth
Distribution.\vspace*{1cm}

\noindent \jelclass{D15, D31, D52, E21, E62, E71, G51}

\thispagestyle{empty} %\addtocounter{page}{-1}
\newpage \addtocounter{page}{-1}

%\setstretch{1.5}
\setstretch{1.25} %\setstretch{1.0}
\setlength{\textwidth}{6.93in}

%%%%%%%%%%%%%%%%%%%%%%%%%%%%
%%%%%%%%%%%%%%%%%%%%%%%%%%%%

\section{Introduction}
\label{sec:introduction}

The marginal propensity to consume (MPC hereafter) is the fraction of a small, one-time, unanticipated windfall that a household spends within a given time period. 
\redGV{The most common definition of an MPC refers to spending over over one quarter (three months) for a windfall of around \$500.}

\redGV{WHAT WE DO: In this paper we offer a systematic analysis of the size of MPCs in the most commonly used class of models of household spending - Precautionary Savings Models. Precautionary Savings Models form the heart of the household side of HA models...}

For several decades, the concept of MPC made only a guest appearance in modern macroeconomics because of the representative agent construct underlying most research. With complete markets (the assumption behind this aggregation result), the aggregate MPC is very small and quantitatively irrelevant. As a result, even in models with nominal rigidities, the much heralded Keynesian multiplier effects are, in fact, negligible. 

In the last two decades, with a strong impetus after the Great Recession, macroeconomics has progressively shifted from representative agent models that focused on the dynamics of aggregate quantities to heterogeneous-agent incomplete-markets models that analyze the dynamics of its entire distribution. In this class of economies, the MPC has regained a central role for two reasons. 

First, because it measures how consumption reacts at impact to a transitory idiosyncratic shock, the MPC is an indicator of the degree of consumption insurance available to households. In heterogeneous-agent models, it is precisely the presence of idiosyncratic risk and liquidity constraints that shapes the distribution of MPCs across households. 

Second, it is by now well understood that what most distinguishes this class of models from their representative agent counterpart are two observations: (i) the aggregate MPC is much larger, and (ii) the distribution of individual MPCs is very heterogeneous. 

When combined with nominal rigidities, the large aggregate MPC implies that the Keynesian multiplier, which determines the general equilibrium propagation of macroeconomic shocks, is substantially stronger. As a consequence, the transmission mechanism of fiscal and monetary policy is being revisited within these models. The main conclusion is that the indirect general equilibrium channels --trivial in the representative agent framework-- here play a prominent role. It follows that to understand the unequal incidence of policies across the income and wealth distribution, one has to go beyond their direct impact and investigate how they affect equilibrium quantities and prices in labor, credit and asset markets.

What is more, the entire distribution of MPCs across households is critical in determining how macro shocks propagate and how effective fiscal and monetary policies can be in stabilizing the economy. Namely, the cross-sectional correlation between a household MPC and its change in income induced by the shock or the policy is a source of amplification (if positive) or dampening (if negative).\footnote{\citet{auclert2018intertemporal} illustrate how in dynamic stochastic general equilibrium New-Keynesian models what matters is not just the contemporaneous MPC, but its the entire time profile, or the intertemporal MPCs. We return on this point later in the paper.} 

A large body of empirical work, surveyed by \citet{jappelli2010consumption}, supports the insight that the distribution of MPCs across households is widespread and that its average is large. Quasi-experimental evidence on economic impact payments indicates that the average quarterly MPC out of windfalls of \$500-\$1,000 is between 15\% and 30\% \citep{johnson2006household,parker2013consumer}. Survey instruments which pose hypothetical questions to households reach similar conclusions \citep{fuster2021would,parker2019reported}. Initially, semi-structural approaches estimated smaller values \citep{blundell2008consumption}, but recent refinement are more in line with other evidence \citep{commault2017does,ganong2020wealth}. These averages mask substantial heterogeneity: the same body of work suggests that many households have MPCs that are close to zero, but some households have MPCs not far from 1 with a great deal of variation in between \citep{gelman2021drives,lewis2019latent}.

From a theoretical perspective, there are a number of reasons why heterogeneous-agent models with uninsurable risk can generate large and heterogeneous MPCs. To begin with, these models feature poor hand-to-mouth (HtM, hereafter) households with zero wealth --or borrowers for which credit constraints bind-- who are eager to consume much of any extra liquidity they receive. Moreover, the two-asset (liquid and illiquid) version of the models also features a second type of constrained agents, the \textit{wealthy} HtM consumers who hold the bulk of their wealth tied up in illiquid form. These agents have consumption responses with respect to small windfalls very much like the canonical HtM households \citep{kaplan2014model,kaplan2014wealthy}.

Uninsurable risk creates two other sources of high MPC. When the utility function displays prudence $(u'''>0)$, the consumption function becomes concave even in the absence of liquidity constraints. Since households want to smooth consumption, they save more `for the rainy days', i.e. future income uncertainty. This precautionary saving motive is declining in wealth, but at a decreasing rate, and thus consumption increases in wealth, but also at a decreasing rate. With occasionally binding liquidity constraints, the consumption function becomes concave even in the absence of prudence for similar reasons. The precautionary saving motive is aimed at avoiding binding constraints which expose the household to consumption fluctuations and welfare losses. In both cases, MPCs can be large for low levels of wealth, also for unconstrained households. \citet{carroll2001theory} and, more recently, \citet{carroll2021liquidity} contain lucid expositions of these arguments and exhaustive surveys of the literature.

Finally, in these models households can differ due ex-ante characteristics, as opposed to random circumstances. Impatience, high willingness to substitute, low returns on saving or behavioral biases can all be sources of high MPCs \citep{aguiar2020hand,carroll2017distribution}. 

\redGV{SHOULD COME EARLIER: Even though the theory behind the MPC is well understood, there is some disagreement and confusion among economists about whether realistically parameterized heterogeneous-agent, incomplete-market models are consistent with this evidence, and about which features of the models are essential for generating MPCs as large and heterogeneous as those in measured in data. The main goal of this paper is to provide clarification on this issue.}

We reach four main conclusions. First, the canonical one-asset incomplete-market model model, calibrated to the observed aggregate wealth-income ratio for the US economy, features an aggregate quarterly MPC is around $4\%$. This value is almost an order of magnitude larger than for the representative agent model, but much still smaller than its empirical counterpart. The key problem is that there is `too much wealth'. This model can generate a quarterly MPC around $20\%$ only when it is parameterized to match an amount of aggregate wealth that is at least 10  times smaller than that of the US. In an alternative interpretation, when it is calibrated to match only mean \textit{liquid wealth} available to US households, thus ignoring the rest of the capital stock. 

Second, extending the model to allow for moderate ex-ante heterogeneity in discount factors, rates of return, or elasticity of intertemporal substitution across households, can yield MPCs of the desired size. The driving force, however, is that households are amassed at the bottom of the distribution, excessively so compared to the data. All these models feature a stark \textit{missing middle} problem in the wealth distribution: a telling model statistic is that the wealth of the median household is 5-10 times smaller than in the data.

Third, we analyze \textit{spender-saver} versions of this model with extreme ex-ante heterogeneity: these economies are populated by one large group with preference parameters similar to agents in the canonical model (savers) and one small group which, because of its innate characteristics, always gravitate near zero wealth (the spenders). This model does not have the missing middle problem and features fairly high MPCs. This model has, however, a host of other problems. First and foremost that the aggregate MPC out of huge windfalls (e.g., \$1M) are identical to the MPC out of small sums, which is implausible and counterfactual \citep{fagereng2019mpc}.

Finally, we show that the two-asset version of the model can resolve this fundamental tension between high MPCs and realistic levels of aggregate wealth by endogenously separating the two assets into a vehicle for long-run saving and one for short-run consumption smoothing. Households in this economy hold most of their wealth in high-return illiquid assets, as in the data. What matters for how aggregate consumption responds to small windfalls of income is, instead, holdings of liquid wealth which are roughly 15\% of the total. 

\redGV{[GV: mention high return gap and that temptation solves that, hopefully. Mention also findings on intertemporal MPC, MPC out of illiquid wealth, size/sign asymmetry, ie last section?]}


\paragraph{Outline}
Section \ref{sec:one_asset} analyzes the canonical one-asset incomplete-markets model. Section \ref{sec:one_asset_extensions} extends the one-asset model model in several directions. Section \ref{sec:two_asset} analyzes the two-asset model. Section \ref{sec:other_mpcs} discusses other concepts of MPCs. Section \ref{sec:conclusions} concludes the paper. The Appendix contains additional details about data, models, calibrations, and simulations.

%%%%%%%%%%%%%%%%%%%%%%%%%%%%%%%%%%%%%%%%%%%%%%%%%%%%%%%
\section{One-Asset Models}\label{sec:one_asset}
In this section, we present findings from various versions of a standard one-asset precautionary savings model. Our one asset models are formulated in discrete time, but we also examine a continuous time version for consistency with the two-asset models in Section XX.


%%%%%%%%%%%%%%%%%%%%%%%%%%%%%%%%%%%%%%%%%%%%%%%%%%%%%%%
\subsection{Baseline One-Asset Model}\label{subsec:baseline_one_asset}

\paragraph{Environment.} \label{sec:one_asset_discrete}
The economy is populated by a measure one continuum of households who survive each period with probability $\left( 1-\delta \right) $. Conditional on surviving, households' discount factors are given by $\tilde{\beta} $,  implying an effective discount factor of $\beta=\tilde{\beta}\left( 1-\delta \right) <1.$ Period utility is given by $u\left( c_{t}\right) ,$ where $u$ is strictly increasing and concave, and $c_{t}$ denotes consumption expenditures. At each date $t,$ households are endowed with labor income $y_{t}$ which follows an exogenous stochastic process described below. Income draws are IID across households. Households can save but not borrow in a risk-free asset $b_{t}$ with rate of return $R=1+r$. The household problem is:
\begin{eqnarray}
\max_{\left\{ c_{t}\right\} } &\mathbb{E}_{0}&\sum_{t=0}^{\infty
}\beta ^{t}u\left( c_{t}\right) 
\label{eq:HP_discrete} \\
&& \text{subject to} \notag \\
c_{t}+b_{t+1} &=& Rb_{t} + y_{t}, \quad b_{t+1} \geq 0, \quad b_0=0 \notag
\end{eqnarray}

The solution to the household problem yields decision rules for consumption  $c\left( b,X\right)$ and next period's wealth $b^{\prime }\left( b,X\right)$, where $X$ denotes the state variables of the income process. These decision rules induce a stationary distribution that we denote by $\mu \left( b,X\right) $, with associated marginal distributions $\mu \left( b,y\right) $ and $\mu \left( b\right) $ over wealth and income, and wealth, respectively.
%\paragraph{Solution.}
%
%In its recursive formulation (spelled out in Appendix \ref{appx:one_asset_baseline}), the household problem has two individual states $\left( b,y\right) \in B\times Y$. Its solution yields decision rules $c\left( b,y\right) $ and $b^{\prime }\left( b,y\right)$, where primes denote next period variables. Combining together the continuum of households in the economy solving this optimization problem, one obtains the stationary distribution $\mu \left( b,y\right) $. Let $\left( \mathcal{B},\mathcal{Y}\right) $ be a subset of the state space $B\times Y$ and $\mathbb{I\in }\left\{ 0,1\right\} $ denote the indicator function. Then, $\mu $ is the fixed point of the functional equation: 
%\begin{equation*}
%\mu _{n+1}\left( \mathcal{B},\mathcal{Y}\right) =\int_{B\times Y}\mathbb{I}%
%_{\left\{ b^{\prime }\left( b,y\right) \in \mathcal{B}\right\} }F\left( 
%\mathcal{Y},y\right) d\mu _{n}\left( b,y\right).
%\end{equation*}%
%


\paragraph{Parameterization.} \label{sec:one_asset_param}
Consistent with our focus on quarterly MPCs, our baseline discrete time model has a period of one quarter. We set $\delta =1/200$ so that the expected adult life span is 50 years. In our baseline parameterization we assume a constant-elasticity utility function $u\left( c\right) =c^{1-\gamma }$ with  $\gamma =1$ so that $u\left( c\right) =\log c$. We set $\underline{b}=0$ so that there is no borrowing and we set the interest rate to $r=0.0025$, or 1\% per year. This partial equilibrium approach keeps our exercise especially clear because it allows us to move the interest rate independently of the discount factor and to highlight their respective importance in determining MPCs.\footnote{In the equilibrium of a closed economy the two would be tightly connected, once a target for the wealth-income ratio is chosen. A number of recent papers in the literature \citep{auclert2018intertemporal,kaplan2018monetary,wolf2019missing} have demonstrated that one can usefully separate macro questions about the transmission of shocks and the effects of policies into (i) partial-equilibrium response and (ii) general-equilibrium amplification. The analysis in this paper is purely about the size of the initial household response to an income shock, and about how different model assumptions matter and why, not about its general equilibrium implications.}

%This preference class exhibits positive third derivative (as long as $\gamma >0$), and hence prudence. 
%We set the credit limit to zero, corresponding to the
%case of 

\paragraph{Income Process.}
We model the process for log income as the sum of two orthogonal components, an AR(1) component and an IID component. We assume that shocks to both components arrive stochastically with a Poisson arrival rate of 1/4, so that shocks are received  on average once a year. Formally: 
\begin{equation}
	z_{t}=\left\{ 
	\begin{array}{ll}
	\phi z_{t-1}+\eta _{t} & \text{with probability }\lambda _{\eta },\quad  \eta _{t}\sim \mathcal{N}\left(-\frac{\sigma^2_{\eta
	}}{2},\sigma^2_{\eta }\right) \label{eq:income_process_discrete} \\ 
	\phi z_{t-1} & \text{with probability }1-\lambda _{\eta }%
	\end{array}
	\right. 
\end{equation}
	and, conditional on a realization of $z_{t},$ 
\begin{equation*}
	\log y_{t}=\left\{ 
	\begin{array}{ll}
	z_{t}+\varepsilon _{t} & \text{with probability }\lambda _{\varepsilon },\quad  \varepsilon _{t} \sim \mathcal{N}\left( -\frac{\sigma^2_{\varepsilon
	}}{2},\sigma^2_{\varepsilon
	}\right)  \\ 
	z_{t} & \text{with probability }1-\lambda _{\varepsilon }.%
	\end{array}%
	\right. 
\end{equation*}
We estimate the parameters of the income process by matching moments of the household labor income distribution from the Panel Study of Income Dynamics. See Appendix \ref{appxsec:PSID} for details.

\paragraph{Wealth Distribution.}
We choose the effective discount factor $\beta$ so that mean wealth in the stationary distribution is consistent with mean wealth in the United States. We express all values as multiples of mean annual household earnings, which we define as labor income plus social security income for retired households. Our income and wealth statistics come from the 2019 Survey of Consumer Finances, from which we exclude households in the the top 5\% of the wealth distribution.\footnote{The top 5\% holds 65\% of the total net worth in the economy. We exclude this group from our calibration because the simple precautionary savings models we consider here are not well suited to explain the top tails of the wealth distribution. \citep{benhabib2018skewed,de2017saving}. } Based on this definition, mean annual earnings is \$67,000 and mean net worth is \$275,000 or 4.1 times mean annual earnings.

%TABLE 1
\begin{table}[ht]
\label{tab:table_baseline}
\input{table_baseline.tex}
\end{table}

%
    \begin{table}[ht] %
    \caption*{Table 1} %
    \centering
    \begin{threeparttable} %
    \begin{tabular}{lccccccc}
    \toprule
    {} &  (1)  &  (2)  &  (3)  &  (4)  &  (5)  &  (6)  &  (7)  \\
     & Data & Baseline & E[a] & Median(a) & E[a] & Median(a) & HtM \\
    \midrule
    Quarterly MPC (\%) &  & 4.6 & 2.7 & 4.3 & 14.0 & 33.7 & 22.0 \\ 
    Annual MPC (\%) &  & 14.6 & 8.5 & 13.6 & 40.8 & 77.4 & 58.7 \\ 
    Quarterly MPC of the HtM (\%) &  & 28.7 & 26.3 & 28.4 & 33.2 & 42.1 & 36.6 \\ 
    Share HtM & 14 & 2 & 2 & 2 & 7 & 37 & 14 \\ 
    Effective discount rate &  & 0.98 & 0.99 & 0.98 & 0.95 & 0.83 & 0.91 \\ 
     & & & & & & & \\ 
    \toprule
    \multicolumn{8}{c}{\textbf{Panel A: Decomposition}} \\
    \midrule
    Gap with Baseline MPC &  &  & -1.9 & -0.3 & 9.4 & 29.1 & 17.4 \\ 
    Effect of MPC Function &  &  & -0.8 & -0.1 & 3.1 & 11.1 & 6.0 \\ 
    Effect of Distribution &  &  & -1.3 & -0.2 & 4.6 & 12.5 & 7.7 \\ 
    \quad Hand-to-mouth &  &  & -0.2 & -0.0 & 1.1 & 6.1 & 2.5 \\ 
    \quad Non-hand-to-mouth &  &  & -1.1 & -0.2 & 3.5 & 6.4 & 5.2 \\ 
    Interaction &  &  & 0.2 & 0.0 & 1.7 & 5.6 & 3.7 \\ 
     & & & & & & & \\ 
    \toprule
    \multicolumn{8}{c}{\textbf{Panel B: Wealth Statistics}} \\
    \midrule
    Mean wealth & 4.1 & 4.1 & 9.4 & 4.6 & 0.6 & 0.1 & 0.3 \\ 
    Median wealth & 1.5 & 1.3 & 3.5 & 1.5 & 0.2 & 0.0 & 0.1 \\ 
    $a \leq \$1000$ & 15 & 3 & 2 & 2 & 6 & 22 & 11 \\ 
    $a \leq \$5000$ & 20 & 12 & 7 & 11 & 29 & 62 & 42 \\ 
    $a \leq \$10000$ & 25 & 18 & 12 & 17 & 44 & 77 & 60 \\ 
    $a \leq \$50000$ & 38 & 40 & 27 & 38 & 79 & 96 & 91 \\ 
    $a \leq \$100000$ & 49 & 52 & 36 & 49 & 90 & 99 & 97 \\ 
    Wealth, top 10\% share &  & 47 & 45 & 46 & 52 & 56 & 52 \\ 
     & & & & & & & \\ 
    \bottomrule
    \end{tabular}
    \end{threeparttable} %
    \label{tab:table_baseline}
    \end{table} %

Table \ref{tab:table_baseline} (Panel B, Column (1)) reports key wealth statistics from the baseline model, with the discount factor chosen to hit this target. The implied annualized value for $\beta$ is $0.98$. Median wealth in the model (1.34) is quite close to its value in the  data (1.54) despite not being explicitly targeted. The baseline model generates fewer very low wealth households than in the data: less than 1\% with zero or negative wealth, compared with 11\% in the data; 2.5\% with less than \$1,000, compared with 15\% in the data; and 12\% with less than \$5,000, compared with 20\% in the data. A common definition of hand-to-mouth households in the literature is households whose wealth is less than half their monthly income \citet{kaplan2014wealthy}. According to this definition, only 2.5\% of households in the model are hand-to-mouth compared with 14\% in the data, because optimizing households seek to save themselves away from hand-to-mouth regions of the asset space. 

%In this class of models, wealth is notoriously less concentrated at the top relative to the data \citep{benhabib2018skewed}. For example, the share held by the top 1\% is 8\% (X\% in the data) and the Gini coefficient is 0.67 (X in the data). These statistics of the top of the wealth distribution are essential in other contexts, but they are not particularly relevant when it comes to the average MPC which is largely determined by the shape of the consumption function and the wealth distribution below median wealth. 

\paragraph{Marginal Propensities to Consume in the Baseline Model}
Our main object of interest is the quarterly MPC out of a one-time unanticipated windfall. For a household with state vector $\left( b,X\right)$ at the time when the windfall is received, the impact MPC (or MPC at horizon $0$) out of a windfall of size $x$ is: 
\begin{equation}
\mathfrak{m}_{0}\left(x;b,X\right) =\frac{c\left( b+x,X\right) -c\left( b,X\right) }{x}.
\label{eq:impactMPC_discrete}
\end{equation}

%
%\begin{equation}
%\mathfrak{m}_{0}\left(x;b,X\right) =\frac{c\left( b+x,X\right) -c\left( b,X\right) }{x}\simeq\frac{\partial c\left( b,X\right) }{\partial b} 
%\label{eq:impactMPC_discrete}
%\end{equation}%
%where the approximation equality holds for $x$ small.
%\footnote{For completeness, we note that the average propensity to consume for this household is $\frac{c(b,y)}{b}$.} 

Our main focus is on MPCs out of \$500 windfalls, which is the approximate size of common stimulus programs from which MPCs are typically measured. In Section \ref{sec:intertemporal_mpc} we explore MPCs out of other size windfalls and losses over other durations. Average MPCs are reported in Panel A of Table \ref{tab:table_baseline}. The average quarterly MPC in the baseline model is $4.6\%$. Figure XYZ displays the MPC as a function of wealth for a household with mean income, $\mathfrak{m}_{0}\left(\$500;b,\overline{X}\right)$, superimposed over the stationary wealth distribution. Note how the MPC quickly converges to the MPC under certainty, $1-\beta$, as wealth rises. At a level of wealth of around $0.5$ (about \$35,000), the effects of concavity of the consumption function on the MPC have already dissipated and it is only households with very low levels of wealth that contribute to generating an MPC that is substantially above $1-\beta$. This is a theme that will resurface as we explore richer versions of the model.


\paragraph{Model Frequency.}
In Appendix \ref{appx:one_asset_baseline} we also report results for the baseline calibration with a model period of one year rather than one quarter, and for a continuous time version of the model. The average annual MPC in the annual model is $14.3\%$, which is essentially the same as the average annual MPC in the quarterly model, which is $14.6\%$.\footnote{This is lower than one would obtain by cumulating the quarterly MPC over four quarters, using the formula $\mathfrak{m}_{4} = \left(1+\mathfrak{m}_{0}\right)^{4} -1$, which is commonly used in the literature. Applying this formula would yield an annual MPC of 19.7\%, 35\% higher than the correct annual MPC}
The quarterly MPC in the continuous time version of the model is 3.0\%, slightly lower than our baseline discrete time model.\footnote{The discrete and continuous time models are not strictly comparable because of necessary differences in the income process. However we note the similar wealth and MPC statistics to ensure that when we move to the continuous time two-asset model in Section XYZ, the differences are not being driven by the switch to continuous time.} 


\paragraph{Decomposition Relative to Certainty Benchmark.} \label{sec:MPC_decomposition_certainty} 
Without income uncertainty or borrowing constraints, the consumption function is linear in wealth, with constant slope $\mathfrak{m}_0^{CE}$, as given by the formula: \footnote{See appendix \ref{appxsec:RA_Model} for the derivation of equation \eqref{eq:mpc_RA}}
\begin{equation}
c_{0}=\mathfrak{m}_{0}^{CE}\left[ Rb_{0}+\sum_{t=0}^{\infty }\left( \frac{%
1 }{R}\right) ^{t}y_{t}\right] ,\text{ with }%
\mathfrak{m}_0^{CE}=1-R^{-1}\left[ R\beta \right] ^{%
\frac{1}{\gamma }}.  \label{eq:mpc_RA}
\end{equation}
We refer to $\mathfrak{m}_0^{CE}$ as the certainty MPC. It is decreasing in the discount factor and, provided that $\beta R<1$, is decreasing in the IES $1/\gamma$. With log-utility $\left( \gamma =1\right)$, the MPC is equal to the effective discount rate, $\mathfrak{m}^{CE}=1-\beta$, which in our baseline calibration would give a quarterly MPC of 0.5\%, nearly one order of magnitude lower than in our baseline one-asset precautionary savings model.

To better understand why the precautionary savings model with uncertainty generates a larger average MPC (4.6\% vs 0.5\%), we propose a decomposition of the difference between the MPCs in the two models into  three sources. First, (i) for those households for whom the borrowing constraint is binding, the  MPC out of small windfalls is equal to 1. For households who are close to the borrowing constraint, the MPC out of a \$500 windfall is typically less than 1, but may still be substantially larger than the certainty MPC. Second, for households away from the borrowing constraint, the MPC can be larger than $\mathfrak{m}^{CE}_0$ because the consumption function is concave. Concavity arises for two reasons: (ii) a future binding borrowing constraint even in the absence of income risk, and (iii) precautionary savings in the face of income risk (due to both prudence and a potentially binding borrowing constraint).

Let $\mathfrak{m}_{0}^{BC}\left( b\right) $ be the MPC function in a model that is identical to the baseline model, except that household income is deterministic and all households receive the average level of income. Despite the absence of income risk, the consumption function in this model is concave because of the presence of a borrowing constraint. The average MPC in the baseline model $\overline{\mathfrak{m}}_{0}$ can then be written as: 
\begin{eqnarray*}
\overline{\mathfrak{m}}_{0} & =& \underset{\text{Certainty}}{\underbrace{\mathfrak{m}_{0}^{CE}}}+\underset{\text{Hand-to-Mouth}}{\underbrace{\int_{\left\{ b\leq\hat{b}\right\} \times X}\left[ \mathfrak{m}_{0}\left(b,X\right) -\mathfrak{m}_{0}^{CE}\right] d\mu \left(0,X\right) }} + \underset{\text{Borrowing Constraints}}{\underbrace{\int_{\left\{ b>\hat{b}\right\} }\left[ \mathfrak{m}_{0}^{BC}\left( b\right) -\mathfrak{m}_{0}^{CE}\right] d\mu \left(b\right) }}  \\
& + &\underset{\text{Precautionary Savings + Income Risk}}{\underbrace{\int_{\left\{ b>0
\right\} \times X}\left[ \mathfrak{m}_{0}\left( b,X\right) -\mathfrak{m}%
_{0}^{BC}\left( b\right) \right] d\mu \left( b,X\right) }}.  \label{eq:decomp_RA}
\end{eqnarray*}
where $\hat{b}$ is the threshold used to define a hand-to-mouth household. The first term is the certainty MPC , the second term captures the role of hand-to-mouth households, the third term captures the role of borrowing constraints absent income risk, and the final term captures the effect of uninsurable risk and precautionary savings.

Table \eqref{eq:decomp_RA} reports this decomposition for the baseline model. The decomposition is somewhat sensitive to the threshold $\hat{b}$ that defines a hand-to-mouth household. With $\hat{b} = \$1,000$, hand-to-mouth households account for 19\% of the difference, borrowing constraints for 53\% and precautionary savings for 28\%. With $\hat{b} = \$3,000$, hand-to-mouth households account for 45\% of the difference, borrowing constraints for 41\% and precautionary savings for 14\%. Thus the decomposition reveals only a minor role for the concavity in the consumption function that is induced by precautionary savings and income uncertainty away from the borrowing constraint . Rather, it is it the borrowing constraint itself, independent of income risk, that is quantitatively more important in accounting for the higher MPC in the one-asset model, both by generating hand-to-mouth households and by generating concavity in the consumption function higher up the wealth distribution. 


%The main reason why the HA model exhibits a higher MPC is because the occasionally binding constraints create a precautionary saving motive that induces concavity in the consumption function near the constraint. The additional concavity associated to prudence and income risk also plays a role, but it is less important (about half) quantitatively. The fact that some households are \emph{currently} constrained (the HtM effect) does not contribute much to the gap because, as explained, this share is small in the model, for example only 2.5\% of households have wealth lower than two weeks of their income. 





%%%%%%%%%%%%%%%%%%%%%%%%%%%%%%%%%%%%%%%%%%%%%%%%%%%%%%%%%%%%%
\subsection{Alternative Calibrations of the Baseline}
\label{sec:one_asset_alternative}

We conduct an extensive sensitivity analysis of the baseline model. Unless  otherwise specified, we always recalibrate the discount factor so that each version of the model has the same ratio of mean wealth to mean earnings of 4.1.


\paragraph{Deviations Between Heterogeneous-Agent Models.}

To understand why different calibrations of the baseline model yield different MPCs, we propose the following decomposition of the difference in average MPCs across models. The average MPC can be larger in one model than in another either because the consumption function is steeper, or because the distribution of households is more concentrated in steeper regions of the state space. Let $\mu^{\ast}$ and $\mathfrak{m}_{0}^{\ast}$ be the distribution and average MPC in the baseline model from Section \ref{subsec:baseline_one_asset}. We can then write the average MPC in an alternative model as:


\begin{eqnarray*} \label{eq:decomp_HA}
\bar{\mathfrak{m}}_{0}&=&\underset{\text{Benchmark MPC}}{\underbrace{\mathfrak{m}_{0}^{\ast}}}+\underset{\text{Consumption Function}}{\underbrace{\int_{B\times \mathcal{X}}\left[ \mathfrak{m}_{0}\left( b,X\right) - \mathfrak{m}_{0}^{\ast}\right] d\mu ^{\ast}\left( b,X\right) }}+\underset{ \text{Distribution}}{\underbrace{\int_{B\times \mathcal{X}}\mathfrak{m}_{0}^{\ast}\left( b,X\right) \left[ d\mu \left( b,X\right) -d\mu ^{\ast}\left( b,X\right) \right] }} \\
&& +\underset{\text{Interaction}}{\underbrace{\int_{B\times \mathcal{X}}\left[ \mathfrak{m}_{0}\left( b,X\right) -\mathfrak{m}_{0}^{\ast}\right] \left[d\mu \left( b,X\right) -d\mu ^{\ast}\left( b,X\right) \right] }}.   
\end{eqnarray*}

The component labeled `Consumption Function' captures the difference in average MPCs that arises because the consumption functions are different in the two models. The component labeled `Distribution' captures the difference in average MPCs  that arises because the stationary distributions of the two models put different mass in different parts of the state space. The component labeled `Interaction' is the component that arises because of the interaction between these two effects.

We then further decompose the Distribution term in \eqref{eq:decomp_HA} into components due to different fractions of hand-to-mouth households in the two models and different distributions of non hand-to-mouth households across the state space:
\begin{equation*}
\underset{\text{Distribution - HtM}}{\underbrace{\int_{\mathcal{X} \times 
\left\{b\leq \hat{b} \right\} }\mathfrak{m}_{0}^{\ast}\left[ d\mu \left( b,X\right) -d\mu ^{\ast}\left( b,X\right) \right] }}+\underset{\text{Distribution - NHtM}}{%
\underbrace{\int_{\left\{ b>\hat{b}\right\} \times \mathcal{X}}\mathfrak{m}%
_{0}^{\ast}\left[ d\mu \left( b,X\right) -d\mu ^{\ast}\left( b,X\right) %
\right] }},
\end{equation*}

%%%%%%%%%%%%%%%%%%%%%%%%%%%%%%%%%%%%%%%%%%%%%%%%%
\paragraph{Target for Wealth-Income Ratio.}

In Table \ref{tab:table_baseline} we report MPCs under alternative aggregate wealth targets. We start by considering a higher target for mean wealth based on the full population without dropping the wealthiest 5\% of households, which is \$750,000, corresponding to a ratio of 9.4 to mean earnings (Column 2). The average MPC under this calibration drops to 2.7\%, driven mostly by the smaller fraction of low wealth households. With this calibration median wealth is 3.5, over twice as large as in the data. Next, we target median wealth instead of mean wealth, which gives an average MPC that is almost unchanged from the baseline because median wealth in the baseline model is already close to the value in the data, despite not being explicitly targeted (Column 3). 

In Columns 4 and 5 we report results from what is known in the literature as a \emph{liquid wealth calibration}. Rather than measuring wealth in the SCF as net worth, we include only liquid wealth, which we define as bank accounts and directly held stocks and bond net of credit card debt.\footnote{Our measure of liquid wealth is defined as (FIN - CDS - SACVBND - CASHLI -OTHMA -RETLIQ) - (OTHLOC + CCBAL + ODEBT).}
The ratio of mean liquid wealth to mean income is only 0.56. The logic behind this calibration is that liquid wealth is a better measure of the wealth that households can readily access to smooth consumption against unexpected income fluctuations. When we target mean liquid wealth, the quarterly MPC is 14\%, much higher than in the baseline and closer to empirical estimates (Column 4). When we target median liquid wealth (\$3,100 or 0.05 times as average earnings), the average quarterly MPC rises to 33\% (Column 5). By targeting a lower amount of aggregate wealth, the liquid wealth calibrations generate large MPCs by making the discount factor significantly lower than in the baseline. The lower discount factor both raises the number of low wealth households, and raises the level of the certainty MPC ($1-\beta$). This suggests that a more direct strategy is to choose the discount factor to directly match the fraction of hand-to-mouth households in the data. When we target 14\% of hand-to-mouth households, the average quarterly MPC is 22\%, in between the mean and median liquid wealth calibrations (Column 6).

The MPC decomposition reveals that in all three of these calibrations, the more concave consumption function and the larger mass of households at low wealth levels both play a role in accounting for the higher MPC. However, quantitatively, it is the distribution, in particular the additional hand-to-mouth households, that plays the  most important role.

Despite the apparent success of liquid wealth calibrations at generating high MPCs, it is important to note that these calibrations are difficult  to integrate into modern dynamic macroeconomic models, because they necessitate abstracting from essentially the entire stock of aggregate assets. This limits the usefulness of these calibrations as a model of the household sector in general equilibrium models with capital (either land, housing or productive capital). For example, the calibration that matches the fraction of hand-to-mouth households in the data effectively abstracts from 93\% of the wealth of the bottom 95\% of households (our baseline sample), or 98\% of total wealth.

%95\% of aggregate assets, and thus restrict the analysis to economies without capital (housing and equity). \redGV{ A `trick' emerged in the literature \citep{auclert2018intertemporal;cui2021quantitative} is that of assuming that the rest of the economy's wealth is locked in extremely illiquid form that is not (or only randomly) accessible to households ---a hybrid between a one-asset model calibrated to liquid wealth and the full-blown two-asset model of Section \ref{sec:two_asset}.}

%%%%%%%%%%%%%%%%%%%%%%%%%%%%%%%%%%%%%%%%%%%%%%%%%
\paragraph{Interest rate}

%TABLE 2
\begin{table}[ht]
\label{tab:table_r_rra}
\input{table_r_rra.tex}
\end{table}

%
    \begin{table}[ht] %
    \caption*{Table 2} %
    \centering
    \begin{threeparttable} %
    \begin{tabular}{lccccc}
    \toprule
    {} &  (1)  &  (2)  &  (3)  &  (4)  &  (5)  \\
     & Baseline & r = 0\% p.a. & r = 5\% p.a. & CRRA 0.5 & CRRA 6 \\
    \midrule
    Quarterly MPC (\%) & 4.6 & 4.4 & 5.2 & 5.3 & 3.0 \\ 
    Annual MPC (\%) & 14.6 & 13.9 & 16.8 & 16.4 & 10.6 \\ 
    Quarterly MPC of the HtM (\%) & 28.7 & 28.5 & 29.5 & 29.0 & 26.6 \\ 
    Share HtM & 2 & 2 & 2 & 3 & 1 \\ 
    Effective discount rate & 0.98 & 0.99 & 0.94 & 0.99 & 0.84 \\ 
     & & & & & \\ 
    \toprule
    \multicolumn{6}{c}{\textbf{Panel A: Decomposition}} \\
    \midrule
    Gap with Baseline MPC &  & -0.2 & 0.6 & 0.7 & -1.6 \\ 
    Effect of MPC Function &  & -0.2 & 0.7 & 0.1 & 0.5 \\ 
    Effect of Distribution &  & -0.0 & -0.1 & 0.6 & -2.1 \\ 
    \quad Hand-to-mouth &  & 0.0 & -0.0 & 0.3 & -0.6 \\ 
    \quad Non-hand-to-mouth &  & -0.0 & -0.1 & 0.3 & -1.5 \\ 
    Interaction &  & -0.0 & 0.0 & 0.0 & 0.1 \\ 
     & & & & & \\ 
    \toprule
    \multicolumn{6}{c}{\textbf{Panel B: Wealth Statistics}} \\
    \midrule
    Mean wealth & 4.1 & 4.1 & 4.1 & 4.1 & 4.1 \\ 
    Median wealth & 1.3 & 1.4 & 1.4 & 1.1 & 2.3 \\ 
    $a \leq \$1000$ & 3 & 3 & 2 & 3 & 1 \\ 
    $a \leq \$5000$ & 12 & 12 & 11 & 14 & 3 \\ 
    $a \leq \$10000$ & 18 & 18 & 18 & 21 & 7 \\ 
    $a \leq \$50000$ & 40 & 40 & 40 & 43 & 25 \\ 
    $a \leq \$100000$ & 52 & 52 & 51 & 54 & 39 \\ 
    Wealth, top 10\% share & 47 & 46 & 46 & 49 & 36 \\ 
     & & & & & \\ 
    \bottomrule
    \end{tabular}
    \end{threeparttable} %
    \label{tab:table_r_rra}
    \end{table} %

Table \ref{tab:table_r_rra} shows that lowering the interest rate from 1\% p.a. to 0\% p.a has a negligible effect on the MPC. Raising the interest rate to 5\% p.a. increases the average quarterly MPC by around half a percentage point. A higher interest rates leads to a calibration with a lower discount factor, which raises the certainty MPC $1-\beta$. This is confirmed by the decomposition which shows that the entirety of the difference in MPC relative to the baseline is due to the consumption function. 

% MPC, but only marginally. The rise is coming entirely from the shape of the consumption function. As clear from \eqref{eq:mpc_RA}, for $\gamma>1$ the MPC is increasing in $R$, so even the consumption function of the incomplete-market model is steeper in wealth.

%%%%%%%%%%%%%%%%%%%%%%%%%%%%%%%%%%%%%%%%%%%%%%%%%
\paragraph{Curvature in Utility}
Table \ref{tab:table_r_rra} also shows that changing the curvature parameter of the CRRA utility function $\gamma$ away from $\gamma=1$ has only a small effect on the average MPC. Higher risk aversion and lower intertemporal elasticity of substitution (higher $\gamma$) strengthens the precautionary savings motive, thus requiring a lower discount factor to generate the same amount of aggregate wealth. There are thus two offsetting forces on the average MPC. First, the lower discount factor raises the MPC at all wealth levels. Second, the stronger precautionary motive means that there are fewer households close to the borrowing constraint in the stationary distribution, for a given aggregate amount of wealth. Table \ref{tab:table_r_rra} shows that for a  coefficient of relative risk aversion of $\gamma=6$, the latter effect dominates and the average quarterly MPC falls from 4.6\% to 3.0\%. The decomposition shows that the 1.4 percentage point lower MPC is comprised of a 0.5 percentage point increase from the lower discount factor, offset by a 2.1 percentage point fall from the smaller fraction of low wealth households. Relative to the baseline, median wealth is almost twice as large in the $\gamma=6$ economy and there only one-third as many households with less than \$1,000. In Section XX we revisit these effects by separating the effects of risk aversion and IES and allowing for heterogeneity.


%%%%%%%%%%%%%%%%%%%%%%%%%%%%%%%%%%%%%%%%%%%%%%%%%
\paragraph{Income Process.}

We also considered several alternative income processes to the one in our baseline model, including : (i) setting $\lambda_{\eta}=\lambda_{\varepsilon}=1$ so that income shocks arrive on average once each quarter, rather than once each year; (ii) estimating the arrival rates $\{\lambda_{\eta}, \lambda_{\varepsilon}\}$ alongside the other parameters by targeting the kurtosis of income growth rates at different lags in addition to the variance of income growth rates; (iii) estimating an annual income process and converting the parameter estimates into quarterly values using the approach in Krueger, Mitman and Perri (201X); (iv) eliminating transitory shocks. We also considered alternative processes in the annual version of the model including: (i) the process in Carrol (1992); (ii) a version with a random walk rather than an AR(1) component; (iii) a version with individual-specific fixed effects.

The results for these models, including the parameters of the income processes can be found in Tables \ref{appxtab:incomeprocess} and XYZ in the Appendix. In all cases, the MPCs are very close to the baseline model. The only version that generates a meaningfully higher MPC is the version without transitory shocks, for which the average quarterly MPC is 5.9\%. There are two offsetting forces. On the one hand, without transitory shocks the precautionary saving motive is weaker and the consumption function is less concave, which lowers the average MPC by 2.3 percentage points. On the other hand, the weaker precautionary motive, leads to a larger fraction of households close to the borrowing constraint in the stationary distribution (17\% of households with less than \$1,000 compared with 3\% in the baseline). This raises the average quarterly MPC by 3.7 percentage points.

%%%%%%%%%%%%%%%%%%%%%%%%%%%%%%%%%%%%%%%%%%%%%%%%%
\paragraph{Survival and Bequests.}

We also examined sensitivity to different assumptions about the survival rate, how the assets of the deceased are distributed and what how the assets of new-born households are determined. As long as the discount factor is always recalibrated to match the same amount of aggregate wealth, none of these assumptions matters. See Tables XYZ in the Appendix.

%%%%%%%%%%%%%%%%%%%%%%%%%%%%%%%%%%%%%%%%%%%%%%%%%
\subsection{Extensions of the One-Asset Model} \label{sec:one_asset_extensions}

%adding ex-ante heterogeneity in preferences and 
%
%
% in discount factors, risk aversion and intertemporal elasticities, and by adding 
%
% heterogeneity in discount factors and rates of return on saving. Next, we generalize household utility. We introduce Epstein-Zin preferences to distinguish between heterogeneity in risk aversion and in the elasticity of intertemporal substitution. Finally, we introduce preferences featuring temptation and self-control, and heterogeneity in the strength of temptation across households. 

%%%%%%%%%%%%%%%%%%%%%%%%%%%%%%%%%%%%%%%%%%%%%%%%%
\subsubsection{Ex-Ante Heterogeneity}\label{sec:one_asset_heterogeneity}

In this section, we extend the one-asset model to allow for various forms of ex-ante heterogeneity.

\paragraph{Heterogeneity in Discount Factors} \label{sec:one_asset_discounting}

We start by allowing for heterogeneity in households' discount factors $\beta$. We consider a a discretized uniform distribution for $\beta $ (annualized) with 5 equally spaced grid points between $[\beta -2\Delta,\beta +2\Delta ]$. We choose the the mid-point $\beta $ to match average wealth of 4.1 as in previous calibrations. Table \ref{tab:beta_r_het} reports results for versions with a moderate amount of heterogeneity ($\Delta = 0.005$, Column 2) and with a large amount of heterogeneity ($\Delta = 0.01$, Column 3). For both calibrations, the median effective discount factor is substantially lower than in the baseline economy. The reason is that in these economies there is a subset of very patient households who have a strong intertemporal savings motive, and it is these households who hold the bulk of aggregate wealth. This can be seen from the much higher share of wealth held by the top 10\% households (88\% and 79\%) compared to the model without discount factor heterogeneity (47\%). The model can therefore match the target for mean wealth while still allowing for a large fraction of households to be impatient. Consequently, the models with discount factor heterogeneity match well the very bottom of the wealth distribution. For example in the calibration with $\Delta=0.01$, 14\% of households are hand-to-mouth, as in the data.

The average quarterly MPC in the models with discount factor heterogeneity are much higher than in the baseline model, and with enough heterogeneity, these models can approach the target empirical values. With $\Delta=0.01$, the average MPC is 19\%, four times as large as the baseline model. While both the shape of the MPC function and the stationary wealth distribution contribute to the higher MPC, the majority of the effect comes from the larger fraction of low wealth households. Figure XYZ displays the MPC functions for high and low $\beta$ households overlaid on the stationary wealth distribution for the two groups of households. The figure illustrates how the impatient households are not only amassed near to the borrowing constraint, but have higher MPCs at all levels of wealth.

We also examine a version of the model in which households switch randomly between different discount factors. We assume that a household draws a new value of $\beta$ with probability $p$, independent of its current value. We set $\Delta=0.01$ and consider annual switching rates of $0.02$, so that the expected duration of a discounting regime has the same expected duration of a lifetime (Column 4); and of $0.1.$, so that the expected duration is equal to a decade  (Column 5). The models with stochastic discount factors generate a lower average MPC compared with a model with the same stationary distribution of $\beta$, but with fixed heterogeneity. The reason is that in the model with stochastic discount factors there is a weaker correlation between wealth and discount factors. In the stochastic $\beta$ model, some low-wealth households who were previously impatient then become patient and quickly accumulate wealth. The wealth distribution is therefore less concentrated at the bottom, which is evident in the much higher value of median wealth. 

However, although the models with discount factor heterogeneity can generate a large average MPC while reproducing the key features of the two tails of the wealth distribution, these models typically fail in reproducing the wealth distribution everywhere in between. For example in the model with $\Delta=0.1$, despite matching the fraction of very low wealth households (14\% hand-to-mouth, 10\% with wealth less than \$1,000) the model features far too many households with wealth just above this threshold: median wealth is ten times smaller than in the data and 75\% of households have wealth below \$50,000, compared with only 38\% in the data. This problem, which we label the `missing middle' is a recurring issue that arises in many of the versions of the model with heterogeneity that we examine below.


%%%%%%%%%%%%%%%%%%%%%%%%%%%%%%%%%%%%%%%%%%%%%%%%%
\paragraph{Heterogeneity in Rates of Return}
\label{sec:one_asset_return}

In the last two columns of Table \ref{tab:beta_r_het}, we report results for an economy with fixed heterogeneity in $r$, uniformly distributed over $[-1\%,1\%,3\%]$ p.a. (Column 6), and over $[ -3\%,1\%,5\%] $  p.a. (Column 7). Heterogeneity in rates of return generates similar results to heterogeneity in discount factors. For example in the calibration in Column 7, 7\% of households are hand-to-mouth, the top 10\% share is 74\% and the aggregate quarterly MPC reaches nearly $12\%$. However these economies also feature the missing middle problem, with 62\% of households having less than \$50,000, which is over one and a half times as many as in the data.

%%%%%%%%%%%%%%%%%%%%%%%%%%%%%%%%%%%%%%%%%%%%%%%%%
\paragraph{Heterogeneity in Risk Aversion and Elasticity of Intertemporal Substitution}
\label{sec:one_asset_ies}

We allow for heterogeneity in the curvature parameter $\gamma$ in the CRRA utility function. The distribution of $\gamma $ is a discretized uniform with 5 geometrically spaced grid points in the interval $[\frac{1}{\bar{\gamma}},\bar{\gamma}]$. In Table XX, we report results for a moderate amount of heterogeneity ($\bar{\gamma} = e^{2}= 7.4$, Column XX) and a large amount of heterogeneity ($\bar{\gamma} = e^{3}=20.1$, Column XX). In both cases we set the mid-point to $\gamma =e^{0}=1$ as in the baseline model. With enough heterogeneity in $\gamma$, these calibrations can generate very large average MPCs, because of the large shares of households in the tails of the wealth distribution. For example the model in Column XX, which has values of $\gamma$ ranging from 0.14 to 7.4, gives an average quarterly MPC of 17\%. Most of the wealth is held by the high risk aversion / low EIS households, and the top 10\% wealth share is around 67\%. The large MPC is driven by the low risk aversion / high EIS households, and 20\% of households are hand-to-mouth. The decomposition reveals that it is indeed this different distribution of households relative to the baseline model, rather than the shape of the MPC function, that is the most important factor in generating a larger MPC. However, like the previous calibrations with heterogeneity, this economy suffers from the same missing middle problem. Median wealth is much lower than in the data \redGV{and virtually every household has wealth lower than \$50,000.}

To distinguish the roles of risk aversion and intertemporal substitution in generating the high degree of wealth concentration and large MPC in this last calibration, we generalize our preferences to those of \citet{epsteinzin}:
\begin{equation}
U_{t}=\left\{ \left( 1-\beta \right) c_{t}^{1-\theta }+\beta \left( \mathbb{E%
}_{t}\left[ U_{t+1}^{\frac{1-\theta }{1-\gamma }}\right] \right) ^{1-\gamma
}\right\} ^{\frac{1}{1-\theta }} \label{eq:EZ_utility}
\end{equation}%
where $\gamma $ is the coefficient of relative risk aversion and $1/\theta $ the IES. 


In Table XYZ we report results for various values of $\gamma$ and $\theta$. Without preference heterogeneity, varying either of these two parameters has only a very small impact on the average MPC once the discount factor is recalibrated to match the same average wealth target. Allowing for heterogeneity in risk aversion also barely has any effect on the average MPC. However, with heterogeneity in the IES, the model is able to generate large average MPCs, suggesting that the results described above with CRRA preferences are being driven by the heterogeneity in the IES rather than in risk aversion. For example, with a geometric uniform distribution of $\theta$ ranging from $\exp{-3}=0.05$ to $\exp{3}=20$, the average quarterly MPC is around 20\%. Households with a high IES are willing to absorb consumption fluctuations, and so hold only small buffer stocks of wealth. Moreover the certainty MPC is given by $\mathfrak{m}_0^{CE}=1-R^{-1}\left[ R\beta \right] ^{\frac{1}{\theta }} \approx 1$ for low $\theta$ (high IES) since $\beta R<1$. So even high IES households with substantial wealth have a large MPC. However, this calibration has 21\% hand-to-mouth households (50\% more than in the data) and suffers from a similar missing middle problem to other versions of the model with heterogeneity. 

%consume immediately a lot out of any additional income. Also in this case, though, the wealth distribution implied by these models severely suffers from the 'missing middle' problem. Under some metric, though, the problem is less serious: comparing XYZ, for given level of median wealth (around 0.3), the economy with heterogeneous EIS features and MPC twice as large as that in the economy with heterogeneous rates of return and heterogeneous discounting, because it is more capable of generating HtM households. 

%%%%%%%%%%%%%%%%%%%%%%%%%%%%%%%%%%%%%%%%%%%%%%%%%
\subsubsection{Behavioral Preferences}

In this section we explore alternative behavioral models for preferences that have been proposed in the literature as ways to generate higher MPCs.


\paragraph{Temptation and Self-control}\label{sec:one_asset_temptation}

\citet{gul2001temptation} propose a model of temptation and self-control, in which consumers are tempted to consume according to a preference specification that overweights current consumption, but can exert some degree of self-control. These preferences have recently gained popularity in the quantitative macro literature \citep{krusell2010temptation,krusell2002time,attanasio2020temptation,nakajima2017assessing, pavoni2017optimal}. We describe the setup formally in Appendix XX. We consider an extreme form of temptation, in which  consumers are tempted to consume according to preferences that places no weight on future consumption ($\beta=0$). The degree of temptation is governed by an additional parameter $\varphi \in [0,\infty ]$. It is straightforward to show that for $\varphi>0$, the temptation and self-control model generates a modified Euler equation in which the effective discount factor depends on the average propensity to consume, which has the effect of making poor households act as if they are more myopic than wealthier households.\footnote{With CRRA preferences, the Euler equation is $u_c(c) = \beta R  \mathbb{E} \left[ \left( 1- \frac{\varphi}{1+\varphi} \left(\frac{c'(x')}{x'} \right)^{\gamma}\right)  u_c(c') \right]$ where $x$ is cash on hand.}


Table \ref{tab:one_asset_temptation} reports results for values of the temptation parameter $\varphi \in \{0.01,0.05,0.07\}$. The calibrations with a high degree of temptation parameter can generate a large MPC without ex-ante heterogeneity, while still matching the target for mean wealth. For example, with $\varphi=0.05$ the average quarterly MPC is 19\%. The decomposition reveals that the shape of the consumption function plays only a minor role and that the large MPC is due to the much more dispersed wealth distribution than in the baseline model. This arises because low-wealth households have a lower effective discount rate than high wealth households, which leads to a large mass of households near the borrowing constraint. In this calibration, 20\% of households are hand-to-mouth, and median wealth is less than one-tenth than in the data. Allowing for heterogeneity in $\varphi$, centered around a median value of $0.05$ has very little effect on either the average MPC or the wealth distribution.

\paragraph{Present Bias} \label{sec:one_asset_IG}

Starting with Laibson (XXX), the other commonly adopted departure from standard preferences is to assume some form of present bias, such as quasi-hyperbolic discounting. We adopt the continuous time formulation as in Laibson, Maxted and Moll (), known as instantaneous gratification. Unlike the model of temptation and self-control, these preferences are not time consistent and so an assumption is needed about whether consumers are naive, meaning that they are unaware that their future selves will have different preferences, or are sophisticated, meaning that they are aware of the time inconsistency and play a game against their future selves. Following Laibson, Maxted and Moll (), and to give the model its best shot at generating a large average MPC, we assume that households are naive.

The instantaneous gratification model features one additional parameter $\delta \leq 1$, which measures the extent of present bias. We describe the setup formally in Appendix XX. When $\delta =1$, households have standard exponential discounting. In Table XXX, we report results for values of $\delta \in \{0.9,0.8,0.7\}$, alongside the baseline continuous time one-asset model. The results suggest that once the instantaneous gratification model is recalibrated to match the same level of aggregate wealth, present bias has a negligible effect on the average MPC. In fact with $\delta=0.7$, the average MPC is \emph{lower} than in the model with exponential discounting. The decomposition reveals that this is because there are two offsetting effects. On the one hand, the instantaneous gratification models has a stationary distribution with more very low wealth households (19\% of households with less than \$1,000, compared with 2.5\% in the baseline). On the other hand, in order to match the same level of aggregate wealth, the calibrated value of $\rho$ is much larger in the model with present bias (0.997 p.a. vs 0.985 p.a.). This means that everywhere above the borrowing constraint the MPC is higher in the model without present bias. Figure XXX illustrates these differences. The decomposition shows that these two effects roughly cancel out.

\subsection{Taking Stock}

When the canonical one-asset heterogeneous agent model is calibrated to match the amount of wealth held by the poorest 95\% of US households, it generates an average quarterly MPC of around $4\%$. This is an order of magnitude larger than the corresponding representative agent model, but still much  smaller than empirical estimates.

The fundamental tension in the baseline model is that it can only generate an average quarterly MPC of around 20\% when it is parameterized to match an amount of aggregate wealth that is $10$ to $20$ times smaller in the US, for example in calibrations based on liquid wealth, rather than total wealth. However these calibrations necessarily ignore almost the entire capital stock and so are of limited use for general equilibrium applications.

When the one-asset model is extended to allow for ex-ante heterogeneity in discount factors, rates of return, or elasticity of intertemporal substitution across, it can generate MPCs of 20\% or higher  in some calibrations. In all these versions of the model, the reason for the larger MPC is that  heterogeneity makes it possible for their stationary distributions to contain a substantial number of hand-to-mouth households, while still generating an average level of wealth as high as in the data. Yet despite sometimes being consistent with fraction of hand-to-mouth households in the data (around 14\%), all these models have far too many households who are not quite poor enough to be hand-to-mouth but still have very little wealth. For example, the fraction of households with less than \$50,000 is typically much higher than in the data and the wealth of the median household is 5 to 10 times smaller than in the data.

\subsection{Spender-Saver Models}

The challenge for one-asset models is therefore to simultaneously generate a large enough amount of wealth in the aggregate as well as a large enough fraction of hand-to-mouth households, while still generating enough households in the middle of the distribution. Before turning to two-asset models, we briefly mention one form of ex-ante heterogeneity in the one-asset model that succeeds in this respect. Inspired by the spender-saver model of Campbell and Mankiw (XX), these are versions of the model with discount factor heterogeneity, in which there are two groups of households, one with a very low discount factor (the spenders) and one with a high discount factor (the savers). Judiciously chosen calibrations of this class of model can match all of these feature of the data.

To illustrate this, Column XX in Table XXX reports results from a calibration in which 15\% of households have a discount factor of 0.4, and the remaining 85\% of households have a discount factor that is chosen so that the ratio of average wealth to average income is 4.1, as in previous versions of the model. In this version of the spender-saver model, 14.8\% of households are hand-to-mouth and the average quarterly MPC of 17\%. Moreover, median wealth is 0.96, which is not too far from its empirical counterpart of 1.54, and around 47\% of households have less than \$50,000 of wealth,  compared with 39\% in the data. There is still a missing middle, but it is nowhere near as extreme as in other versions of the one-asset model with heterogeneity.

Nonetheless, despite its success at generating a large average MPC and matching the wealth distribution, the spender-saver model has some important drawbacks that limit is usefulness for counterfactual policy analysis. Most importantly, because the large MPC is driven entirely by a group of very impatient households, this model features a similar sized MPC out of much larger windfalls, and an intertemporal MPC function that drops off sharply after the first quarter. We return to these comparisons in Section XX. In addition, the spender-saver model implies a weaker connection between low income and low wealth than in reality, because most of the hand-to-mouth households have low wealth because of a strong preference for spending rather than because of low income reaslizations.

%%%%%%%%%%%%%%%%%%%%%%%%%%%%%%%%%%%%%%
\section{Two-Asset Models} \label{sec:two_asset} 

In this section, we extend the precautionary savings model to include two assets: a liquid asset with low return and an illiquid asset with higher return but which is subject to adjustment frictions. We demonstrate that the two-asset model can resolve the intrinsic tension in the one-asset model even in the absence of ex-ante heterogeneity.

In two-asset heterogenous agent models, households can separate their different savings motives into distinct assets. Precautionary and smoothing motives against small, regular income fluctuations induce households to accumulate a buffer of liquid assets, but since the return on this asset is low, a negative intertemporal motive prevents households from accumulating large amounts of liquid wealth. The bulk of household savings is done in the high return illiquid asset, motivated both by a positive  intertemporal motive and precautionary and smoothing motives against large, infrequent income fluctuations. At endogenous intervals, households move funds between their liquid and illiquid accounts as desired. For example, in response to a large negative income shock, a household who has exhausted their buffer of liquid wealth can pay a fee or exert effort to withdraw funds from their illiquid account. Similarly, a household who has experienced a long stream of positive income growth and has accumulated excess liquid wealth can transfer funds to their illiquid account, which pays a higher return and is therefore a better vehicle for long-run saving. 

Since most households end up holding the vast majority of their wealth as illiquid assets, which cannot be  used for short-term consumption smoothing, they expose themselves to potentially larger consumption fluctuations than if they saved only in liquid assets. However, the welfare cost of these fluctuations is second-order, relative to the first-order gain from earning a higher return on illiquid savings. Because of this trade-off, the model generates \emph{wealthy} hand-to-mouth households, who have positive, and sometimes substantial, holdings of illiquid wealth, but very little liquid wealth. Wealthy hand-to-mouth households co-exist alongside the traditional \emph{poor} hand-to-mouth households, who hold very little wealth, either liquid or illiquid. Adopting the same definition of a hand-to-mouth household as in previous sections (less than half of monthly income), 27\% of households were wealthy hand-to-mouth in the 2019 SCF, in addition to the 14\% of poor hand-to-mouth households. Therefore, in total 41\% of US households are hand-to-mouth. Since wealthy hand-to-mouth households also have a large MPC out of small one-time windfalls in the two-asset model, it is the presence of this additional group of hand-to-mouth households that enables the two-asset model to generate a large average MPC while remaining consistent with the distributions of liquid and illiquid wealth in the data.


%%%%%%%%%%%%%%%%%%%%%%%%%%%%%%%%%%%%%%
\subsection{Baseline Two Asset Model} \label{sec:two_asset_HP}

\paragraph{Environment}
We write the two-asset model in continuous time. The economy is populated by a continuum of households who discount the future at the effective rate $\rho=\tilde{\rho}+\delta<1$ where $\tilde{\rho}$ is the rate of time preference and $\delta$ is the death rate. Flow utility is given by $u\left( c_{t}\right) ,$ where $u$ is strictly increasing and concave, and $c_{t}$ denotes consumption expenditures. At each date $t,$ households are endowed with log labor income $y_{t}\in Y$, which follows an exogenous stochastic process described in Appendix XYZ. Households can save but not borrow in two assets: (i) a liquid asset $b$ with return $r^{b}$; and (ii) an illiquid asset $a$ with return $r^{a}>r^{b}$. At rate $\chi$, households receive an opportunity to rebalance their financial portfolio which they can take by paying a fixed transaction cost $\kappa$. Between rebalance dates, the illiquid asset accumulates in the background and the household solves a standard consumption-savings problem out of their liquid assets. The HJB for the household problem is:
\begin{eqnarray*}
\rho v\left(a,b,X\right)&=&\max_{c} \quad u\left(c\right)+v_b\left(a,b,X\right)\dot{b} + v_a\left(a,b,X\right)\dot{a} + \mathcal{A} v\left(a,b,X\right)\label{eq:HP_2A_recursive}  \\
 && +\chi\left[v^{*}\left(a,b,X\right)-v\left(a,b,X\right)\right]  \\
 && \text{subject to}  \\
\dot{b} &=& r^{b}b+y(X)-c, \quad b\geq 0  \\
\dot{a} &=& r^{a}a, \quad a \geq 0 
\end{eqnarray*}
where $X$ is the vector of state variables for income and $\mathcal{A}$ is the infinitesimal generator of the stochastic process for income. The third term in the HJB equation describes the value of rebalancing the asset portfolio. The value function after rebalancing, $v^{*}\left(a,b,y\right)$, is defined by
\begin{eqnarray*}
v^{*}\left(a,b,X\right) &=& \max \left\{ \omega\left(a+b,X\right),v\left(a,b,X\right)\right\} \label{eq:HP_2A_rebalance} \\ 
&& \text{where} \\
\omega\left(a+b,y\right) &=& \max_{a',b'} v\left(a',b',y\right)  \\
 && \text{subject to}  \\
a'+b' & \leq & a+b-\kappa, \quad a', b' \geq 0
\end{eqnarray*}
Upon receipt of an adjustment opportunity, a household will choose not to rebalance their portfolio if the transaction cost $\kappa$ exceeds the gains from rebalancing. If the household chooses to rebalance, it can choose any feasible combination of liquid and illiquid assets 

The solution to this problem yields decision rules decision rules for consumption $c\left( a,b,X\right)$, and for the optimal rebalanced portfolio $a^{\prime }\left( a,b,y\right) $ and $b^{\prime }\left( a,b,y\right)$, as well as the stationary distribution of households $\mu \left( a,b,X\right) $. 

\paragraph{Parameterization} \label{sec:two_asset_param}
Relative to the continuous time one-asset model, there are three additional parameters: (i) the arrival rate for rebalancing opportunities $\chi$; (ii) illiquid asset return $r^a$; and (iii) the transaction cost $\kappa$. In our baseline model we set $\chi=1$ so that households get opportunity to rebalance on average once per quarter. We then set $r^a$ and $\kappa$, along with the liquid return $r^b$ and the discount rate $\rho$ to match four targets: (i) a ratio of mean total wealth to mean earnings of 4.1, as in the one-asset model; (ii) a ratio of mean liquid wealth to mean earnings of 0.56; (iii) a total share hand-to-mouth households (both wealth and poor) of 41\%;  and (iv) a share of poor hand-to-mouth households of 14\%. 

%All four parameters in equilibrium impact all four targets, but intuitively a smaller rate of time preference increases total wealth holdings; the costlier is rebalancing, for given frequency, the less likely are households to shift funds between liquid and illiquid wealth, which raises mean liquid wealth; a smaller liquid rate $r^b$ curtails incentive to save for poor households without illiquid wealth and makes them more likely to be HtM; a wider gap between the liquid and illiquid rate makes it easier for the model to generate wealthy HtM households who are willing to bear the cost of income fluctuations in order to earn a higher return on their savings.

Table \ref{tab:two_asset_baseline} reports the calibrated parameters and wealth statistics in our baseline two asset model. The model matches well the targets for total wealth and the shares of poor and wealthy hand-to-mouth are closely matched. To achieve this, the calibration requires a substantial gab between the liquid and illiquid returns. But the model is not able to generate enough liquid wealth, while still matching the hand-to-mouth targets. Mean liquid wealth in the data is roughly half as large as the target (0.25 vs 0.56). Median liquid wealth, however, in the model is close to its value in the data, around 5\% of average annual labor income. We return to these two features of the calibration at the end of this section. The calibrated transaction cost is around \$500.\footnote{The combination of transaction frequency and cost is not binding: every quarter 13\% of the population chooses to rebalance.}  

\paragraph{Marginal Propensities to Consume in the Two Asset Model} \label{sec:two_asset_results}
The average quarterly MPC in the baseline two-asset model model is 17\%, around 6 times larger than the corresponding average MPC in the continuous time one-asset model (Table \ref{tab:two_asset_baseline}, Columns 1 and 2). A simple back-of-the-envelope calculation confirms that if we multiply the average MPC for poor hand-to-mouth households (25\%) by their share (12\%), the average MPC for wealthy hand-to-mouth households (28\%) by their share (29\%) and the average MPC for non hand-to-mouth households (10\%) by their share (59\%), and then adds up, we obtain an average MPC of 17\% as expected.

%The two-asset model is able to generate such a high MPC while at the same time hitting easily the aggregate wealth target. The reason is that households in this economy hold most of their wealth in illiquid assets, as in the data. What matters for how aggregate consumption responds to small windfalls of income is, largely, holdings of liquid wealth which are roughly 15\% of the total. This portfolio configuration is in line with US data where the bulk of wealth is held in housing and retirement accounts, like 401(k), both rather illiquid saving vehicles. Also consistently with the data, 40\% of households in the two-asset model are HtM in terms of liquid wealth, and are therefore very responsive to income shocks. Thus, the two-asset approach can resolve the fundamental tension in the canonical model between high MPCs and realistic levels of aggregate wealth.

To better understand the source of the higher average MPC in the two-asset model than the one-asset model, we apply the decomposition in \eqref{eq:decomp_HA} by computing the average MPC as a function of net worth in the two asset model, $\mathfrak{m}_0(w,y)=\int_{a+b=w}\mathfrak{m}_0(a,b,X)d\mu(a,b,X)$. While both the shape of the consumption function and the distribution of households play a role, the MPC function is quantitatively more important for application than in the comparisons across one-asset models. The reason is that even at moderate levels of wealth there are some low liquid wealth households, which generates an MPC function that declines more slowly with wealth than in the one asset model. This feature of the two-asset model is illustrated in Figure XYZ, which shows the average MPCs as a function of net worth in both models, overlaid on their stationary distributions.


The statistics on the wealth distribution in Table \ref{tab:two_asset_baseline} also reveal that the two-asset model does not feature the missing middle problem that was a feature of the one-asset models with ex-ante heterogeneity. Median net worth is near 1, which is much closer to the empirical target of 1.54, and the share of households with wealth less than \$50,000 is 47\%, compared with 38\% in the data. The reason is that in the two-asset models the bulk of the wealth of households in the middle of the distribution is held in illiquid assets, which has only a small effect on their MPCs. To illustrate this point, Figure XYZ shows that the average MPC is essentially flat when expressed as a function of illiquid assets.

%%%%%%%%%%%%%%%%%%%%%%%%%%%%%%%%%%%%%%%%%%%%%%%%%%%%%%%%%%
\subsection{Alternative Calibrations of Two Asset Model}
We examine the robustness of our finding for the two-asset model with respect to some of the key parameters of the model. In these experiments, whenever we modify one parameters, we recalibrate only the discount rate $\rho$ to match the same target for total wealth, unless otherwise noted. The decompositions in this section are relative to the baseline two-asset model. 

\paragraph{Rebalancing Frequency.}
Increasing or decreasing the frequency of rebalancing opportunities from an average of once per quarter to an average of once per month (Table \ref{tab:two_asset_baseline}, Column 3) or once per year (Column 4) has only a small effect on the average MPC. However the calibration with less frequent rebalancing opportunities is better able to match the mean level of liquid wealth. Households hold a larger buffer of liquid wealth both because opportunities to rebalance arrive less frequently and because households that are accumulating wealth must hold their savings in the liquid account for a longer duration on average before transferring them to their illiquid account. This model also features a slightly larger average MPC despite having fewer hand-to-mouth households because the  less frequent adjustment opportunities means that the wealthy hand-to-mouth households who are waiting to withdraw resource have a higher MPC than in the baseline calibration.
\redGV{I re-wrote. I think this is correct but not sure}
% augments the mean liquid wealth gap between model and data. Instead, making the rebalancing opportunity more infrequent (once a year instead of once a quarter) allows the model to nearly hit this target. Households who receive positive income shocks save into liquid wealth in order to transfer some of it to the illiquid account. If the rebalancing opportunity does not materialize, they keep accumulating liquid wealth, so it's easier for this economy to achieve high mean liquid wealth to income ratios. This model has an even larger MPC, above 18\%. The reason, according to our decomposition, is the steeper consumption function whereas the different wealth distribution, as expected, has a small negative effect. These households with `excess liquid saving' have very small gains from saving an extra dollar in liquid wealth, so when they receive an unexpected income windfall they consume more of it. \redGV{[GV: is this the right explanation? I think there is also the fact that knowing they can withdraw less frequently if needed, they keep more in liquid wealth]}. 

\paragraph{Rebalancing Cost.}
Varying the transaction cost from \$250 to \$2,000, while keeping the arrival rate of rebalancing opportunities at its baseline value of 1.0 has only very small effects on the average MPC because there are offsetting effects on the distribution and the shape of the consumption function. See Table XYZ in Appendix XYZ.

\paragraph{Rates of Return.}
Changing rates of return on liquid and illiquid wealth has, instead, more of an impact. Raising the liquid rate or reducing the illiquid rate narrows the return gap between the two assets and, as explained, fewer HtM households emerge in equilibrium which pushes the MPC down. The decomposition confirms that the reason why the MPC are lower in these cases is because there are fewer HtM households with large MPCs.


%%%%%%%%%%%%%%%%%%%%%%%%%%%%%%%%%%%%%%%%%%%%%%%%%%%%%%%%%%
\section{Other Concepts of MPCs}


\redGV{THIS SHOULD GO LATER
To compute the MPC at different horizons, we proceed as follows. Let, for example, $t=1$ be the horizon of interest. Then, the MPC at horizon $1$ out of a windfall income $x$ is: 
\begin{equation*}
\mathfrak{m}_{1}\left(x; b,y\right) =\frac{\int_{Y}\left[ c\left( b^{\prime
}\left( b+x,y\right) ,y^{\prime }\right) -c\left( b^{\prime }\left(
b,y\right) ,y^{\prime }\right) \right] dF\left( y^{\prime },y\right) }{x}
\label{eq:horizon1MPC_discrete}
\end{equation*}%
Iterating this procedure forward, one obtains $\mathfrak{m}_{t}\left(x;
b,y\right) $, for all $t>0.$ 
%For example, 
%\begin{equation*}
%\mathfrak{m}_{2}\left( b,y\right) =\frac{\int_{Y}\int_{Y}\left[ c\left(
%b^{\prime \prime }\left( b^{\prime }\left( b+x,y\right) ,y^{\prime }\right)
%,y^{\prime \prime }\right) -c\left( b^{\prime \prime }\left( b^{\prime
%}\left( b,y\right) ,y^{\prime }\right) ,y^{\prime \prime }\right) \right]
%dF\left( y^{\prime \prime },y^{\prime }\right) dF\left( y^{\prime },y\right) 
%}{x},
%\end{equation*}%
%and so on. 
The cumulative MPC until horizon $T$ is simply the
sum of the MPCs at each horizon $t=0,1,..., T.$ The average (or aggregate) MPC at horizon $t$ is obtained by integrating the function $\mathfrak{m}_{t}\left(x; b,y\right) $ under the stationary distribution, i.e. 
\begin{equation}
\bar{\mathfrak{m}}_{t}(x) =\int_{B\times Y}\mathfrak{m}%
_{t}\left(x;b,y\right) d\mu \left( b,y\right) .  \label{eq:averageMPC}
\end{equation}
Finally, we are also interested in the MPC out of the news that a windfall of size $x$ will be received in the future. For example, the MPC at horizon $-1$, i.e. out of the announcement that $x$ will be paid next period, is:
\begin{equation}
\mathfrak{m}_{-1}\left(x;b,y\right) =\frac{c\left(x; b,y\right) -c\left( b,y\right) }{x}
\label{eq:newsMPC_discrete}
\end{equation}%
where  $c(b,y)$ is the solution to the Bellman equation corresponding to the optimization problem \eqref{eq:HP_discrete}:
\begin{eqnarray*}
v(b,y) &=& \max_{c} u(c) + \beta \mathbb{E}\left[ v(b',y')|y\right]
\end{eqnarray*}
and $c(x;b,y)$ is the solution to the following Bellman equation, modified to account for the fact that the household expects $x$ next period:
\begin{eqnarray}
v(b,y) &=& \max_{c} u(c) + \beta \mathbb{E}\left[ v(b'+x,y')|y\right]
\label{eq:HP_discrete_recursive_news}
\end{eqnarray}
and subject to the same set of constraints as \eqref{eq:HP_discrete_recursive}.
Finally, we note that we will compute MPCs for different values of $x$, both positive and negative.}

\label{sec:other_mpcs}
So far, we focused on the quarterly MPC at impact out of \$500. In this section, we extend the analysis to (i) MPCs out of unexpected income shocks of different sizes, both positive and negative, (ii) MPCs at different horizons, including MPCs out of the news of future windfalls, and (ii) MPCs out of unexpected illiquid injections (or capital losses) in the two asset models.

\subsection{Sign and Size Asymmetries}
\label{sec:other_mpcs_size}
To explore sign and size asymmetries, we compute MPC out of +/- \$1, \$500, \$5,000. To summarize the results, in the left-panel of Figure \ref{fig:MPC_asymmetry} we plot these MPCs for a number of models.

Because of the concavity in the consumption function, the MPC out of small income changes is higher than for larger ones. However, these size asymmetries are not major. In all cases when we increase the windfall by a factor of 10 (from \$500 to \$5,000), the MPC only drops by roughly 15-20\%. Concavity also implies that MPC out of income losses are larger than for income gains of the same absolute magnitude, and that size-asymmetry is reversed for negative windfalls: the MPC out of large income losses is higher than for small ones. 

The reason why, at least in one-asset models, these effects are not very strong is evident from the right-panel of Figure \ref{fig:MPC_asymmetry} which plots the MPCs out of these various income changes as a function of wealth. The asymmetries we discussed are significant only in the region near the borrowing limit, hence for a relatively small share of households in the economy.

Interestingly, the two asset model features much stronger size asymmetry for income losses, a reflection of the fact that many more of its households are HtM (in liquid wealth). \redGV{[GV: it would be nice to show that as the income changes grow, more households adjust and that reduces the size of the MPC. Ideas?]}

Finally, we note that we have chosen the fraction spent out of a fixed dollar amount (\$500) as opposed to a common fraction of one's own income as our MPC baseline for two reasons: because there exists quasi-experimental evidence which can be used to assess different models and because, as we just documented, there are size-asymmetries. 

The MPC out of an unexpected one-period percentage change in income, common across households, is also arguably an interesting concept as it describes the partial-equilibrium impact of an idealized transitory aggregate shock with even incidence across all households. For ease of comparison, we computed this magnitude for a percentage change in income that, on average, equals \$500 in the population (i.e. it is larger for high-income households and smaller for low-income ones). \redGV{[GV: Add results]}. 

\subsection{Intertemporal MPCs}
\label{sec:other_mpcs_intertemporal}
We now use the model to derive the entire intertemporal profile of MPCs from horizon -4 (i.e. a year before the shock) to +4 (i.e. a year after the shock). The quarterly MPC out of of \$500 at horizon $-t$ should be interpreted as the share consumed by household upon receiving the news that a windfall will occur in $t$ quarters. The MPC at horizon $t$ is the fraction consumed $t$ quarters after receipt of an unexpected windfall.

Figure \ref{fig:MPC_intertemporal} reports the time profile of all these MPCs across several models. In the one-asset baseline the profile is very flat because the share of constrained households is quite small and thus, in the aggregate, intertemporal consumption smoothing is effective. 

In the one-asset model calibrated to the share of HtM the profile is instead a lot more cusp-shaped. The MPC out of the news is much smaller than the contemporaneous MPC, and, at the same time, the contemporaneous MPC is much bigger than the MPC a year after receipt of the extra income. The models with discount factor heterogeneity, EIS heterogeneity and the temptation and self-control model display similar shapes for the same reason.

The two-asset model XYZ

\subsection{MPC out of Illiquid Wealth} 
\label{sec:other_mpcs_illiquid}
In the two-asset model, we can also compute the MPC out of a small injection of illiquid wealth. Conceptually, the closest empirical counterpart of this MPC would be an estimated MPC out of changes in housing wealth or out of unrealized capital gains/losses on stocks. To put our analysis in the context of recent empirical contributions, we note that \citet{mian2013household} estimate an average annual MPC out of housing wealth between 5 and 7 percent, and uncover a great deal of heterogeneity with respect to income and leverage.\citet{di2020stock} estimate MPCs out of unrealized capital gains in stockmarket wealth which vary between 20\% per year for the bottom half of the distribution and 3\% for the top third. 

Figure \ref{fig:MPC_illiquid} plots the MPC out of an unexpected injection of illiquid wealth equal to \$500 and \$5,000 as a function of liquid wealth holdings. The average MPCs are, respectively, 2\% and 2.5\%, but there is a great deal of heterogeneity across the wealth distribution. Note that the size asymmetry is reversed for illiquid windfalls: large gains lead to a bigger MPC. The reason is that, upon a large gain, the household is more likely to pay the transaction cost (recall that its is estimated to be around \$500) and realize part of that capital gain.  

Interestingly, the MPC out of unexpected capital losses of the same sizes are much larger, around 18\%. \redGV{[GV: possibly because households want to rebuild their illiquid wealth target and save more?]} 

\section{Conclusions}
\label{sec:conclusions}

\redGV{MATERIAL FOR TEMPTATION APPENDIX}

is the one introduced by 
 There, consumers suffer from temptation but, at a
cost, are able to impose some degree of self-control in the face of this
temptation. These axiomatically founded preferences can be represented by a
pair functions: a standard period-felicity function $u$, and a function $\omega$
which indexes the value of the tempting allocation. The household
problem can be written compactly in recursive form as:%
\begin{equation*}
v\left( b,y\right) =\max_{b^{\prime }\geq 0}\left\{ u\left( Rb+y-b^{\prime
}\right) +\varphi \omega\left( b,y,b^{\prime }\right) +\beta \mathbb{E}%
v\left( b^{\prime },y^{\prime }\right) \right\} -\varphi \max_{\tilde{b}%
^{\prime }\geq 0}\omega\left( b,y,\tilde{b}^{\prime }\right) .
\end{equation*}%
The parameter $\varphi \geq 0$ measures the strength of the temptation. When 
$\varphi =0,$ we are back to the standard model. The higher $\varphi $ is,
the larger the utility from \textquotedblleft going on a
splurge\textquotedblright\ today. The continuation value from the tempting
outcome can be written generally as 
\begin{equation*}
\omega\left( b,y,\tilde{b}^{\prime }\right) =u\left( Rb+y-\tilde{b}^{\prime
}\right) +\beta \alpha v\left( \tilde{b}^{\prime },y^{\prime }\right) .
\end{equation*}%
To simplify the analysis, we set $\alpha =0,$ i.e. all the gains from the
tempting allocation accrue in the current period, which leads to the
formulation 
\begin{equation}
v\left( b,y\right) =\max_{b^{\prime }\geq 0}\left\{ \left( 1+\varphi \right)
u\left( Rb+y-b^{\prime }\right) +\beta \mathbb{E}v\left( b^{\prime
},y^{\prime }\right) \right\} -\varphi u\left( Rb+y\right)
\label{eq:TSC_utility}
\end{equation}%
where the last terms reflects the fact that there is no gain in smoothing
the tempting allocation over time. The first-order condition of this problem
is 
\begin{equation}
u_{c}\left( c\right) =\beta R\mathbb{E}\left[ \left( 1-\frac{\varphi }{%
1+\varphi }\frac{u_{c}\left( Rb^{\prime }+y^{\prime }\right) }{u_{c}\left(
c^{\prime }\right) }\right) u_{c}\left( c^{\prime }\right) \right] .
\label{eq:TSC_FOC}
\end{equation}%
This first-order condition can be interpreted as a modified Euler equation,
with an endogenous discount factor. Note that, for given $\varphi $, wealthier individuals are subject to weaker temptation because the ratio of marginal utilities is lower. Discounting decreases with wealth from $\beta$ for the very rich to $\beta \frac{1}{1+\varphi}$ for households who consume nearly all their cash in hand.

\end{document}
%%%%%%%%%%%%%%%%%%%%%%%
%%BIBLIO
%%%%%%%%%%%%%%%%%%%%%%%
%\vspace*{2cm}
\renewcommand{\baselinestretch}{1.04} 
{\small
\bibliographystyle{econometrica}
\bibliography{AR_bibtex}
}

%%%%%%%%%%%%%%%%%%%%%%%%
%%APPENDIX
%%%%%%%%%%%%%%%%%%%%%%%%%
\newpage
\setcounter{page}{1}
\clearpage\newpage
\input{kv_annualreviews_mpc_appendix.tex}

\end{document}
